\chapter{NANOPARTICLES}
Single-domain particle with uniaxial anisotropy 

Bulk: $\theta =0, \theta = T$ V is large

Nano:  $k_B T$ becomes comparable in order of magnitude to $K V$
fluctuation between $\theta = 0$ and $\theta = \pi$

$\theta$ is the inclination of the Magnetisation according to the easy axis of the material (geometry)

Imagine plot $E_{ani} = KVsin^2\theta$

Nanoparticle "Superspin" occures for $\mu \approx 10^3$ till $10^5 \mu_B$

$\tau	= \tau_0 exp(\frac{KV}{k_BT})$ for $k_BT \leq KV$

$\tau = 10^{-11}s, tau_0 = 10^{-9 \text{till} -11}$

$K = 10^6 Jm^{-3}$

$T = 300K$

So the size of the particle changes the relaxation time

a$d = 6nm$ gives about $540years$ and $d = 12nm$ gives $\approx$ age of universe

Whether a nanoparticle appear superparamagnetic depends on measuring time:

short time scale: $ M  = M$
long time scale: $ M  = 0$

Influence by magnetic field:

$E = KVsin^2\theta - \hat{\mu} \hat{B}$
like paramagnet, but $\mu$ is large

$M  \approx M_0 L (\frac{mB}{k_BT})$
$L(x) = coth(x - \frac{I}{x})$

$\chi = \frac{\mu_0n\mu^2}{3k_BT}$

We should know now the difference among paramagnetism and superparamagnetism

superparamagentic materials have orderd magnetic structure inside! and don't show hysteresis

The temperature where our material becomes superparamagnetic from a paramagnetic state is called the blocking temperature $T_B$:

Spin dynamics below $T_B$ ("$\bar{q} = 0 spinwaves$"):



